The output of the aes.c program matches the expected output. 
This indicates that the AES implementation is correct.

One possible addition to this AES implementation is to integrate integrity checks.
This can be done by computing the SHA-256 hash of the plaintext and including it
as a header of the plaintext before encrypting the entire message. 
The receiver can then compute the SHA-256 hash of
the decrypted plaintext without the header and compare it to the included hash header.
If the hashes match, the integrity of the message is verified.

The decrypted output of the ciphertext output of the rsa.c program matches the original
plaintext input. This indicates that the RSA implementation is correct.

To implement integrity checking for RSA, the sender can compute the SHA-256 hash of the
plaintext and encrypt it with the private key. The receiver can then decrypt the hash
with the public key and compare it to the SHA-256 hash of the plaintext. If the hashes
match, the integrity of the message is verified.