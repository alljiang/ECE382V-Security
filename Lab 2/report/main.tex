%%
%% This is file `sample-authordraft.tex',
%% generated with the docstrip utility.
%%
%% The original source files were:
%%
%% samples.dtx  (with options: `authordraft')
%% 
%% IMPORTANT NOTICE:
%% 
%% For the copyright see the source file.
%% 
%% Any modified versions of this file must be renamed
%% with new filenames distinct from sample-authordraft.tex.
%% 
%% For distribution of the original source see the terms
%% for copying and modification in the file samples.dtx.
%% 
%% This generated file may be distributed as long as the
%% original source files, as listed above, are part of the
%% same distribution. (The sources need not necessarily be
%% in the same archive or directory.)
%%
%% The first command in your LaTeX source must be the \documentclass command.
\documentclass[sigplan,screen]{acmart}

\usepackage{xurl}
\usepackage{xspace}
\usepackage{cancel}
\usepackage{pifont}
\usepackage{comment}
\usepackage{listings}
%\usepackage{subfig}
\usepackage{multirow}
\usepackage{booktabs}
\usepackage[boxed,lined,noend]{algorithm2e}
\usepackage[dvipsnames]{xcolor}
\usepackage{enumitem}
\newcommand{\para}[1]{\vspace{.05in} \noindent \textbf{#1}}
\usepackage{fancyhdr}
\usepackage[normalem]{ulem}
\usepackage{flushend}
\usepackage{soul}
\usepackage{tikz}
\usepackage{cleveref}
\usepackage{graphicx}
\usepackage{upquote}
\usepackage{xcolor,beramono}
\usepackage{microtype}
\usepackage[normalem]{ulem}
\usepackage{subcaption}
\usepackage{epsfig}


\lstset{frame=tb,
  language=C,
  %aboveskip=3mm,
  %belowskip=3mm,
  numbers=left,
  stepnumber=1,
  xleftmargin=1cm,xrightmargin=-1cm,numbersep=2pt,
  showstringspaces=false,
  columns=flexible,
  escapechar=|,mathescape,
  basicstyle={\normalsize\ttfamily},
  numberstyle=\tiny\color{gray},
  keywordstyle=\color{blue},
  commentstyle=\color{ForestGreen},
  stringstyle=\color{mauve},
  breaklines=true,
  morekeywords={impIf},
  breakatwhitespace=true,
  tabsize=3,
  frame=none,
  moredelim=**[is][\color{ForestGreen}]{@}{@},
  moredelim=**[is][{\btHL[fill=white!10,draw=red,solid,thin]}]{`}{`},
}



\makeatletter

%%
%% The majority of ACM publications use numbered citations and
%% references.  The command \citestyle{authoryear} switches to the
%% "author year" style.
%%
%% If you are preparing content for an event
%% sponsored by ACM SIGGRAPH, you must use the "author year" style of
%% citations and references.
%% Uncommenting
%% the next command will enable that style.
%%\citestyle{acmauthoryear}

%%
%% end of the preamble, start of the body of the document source.
\begin{document}

%%
%% The "title" command has an optional parameter,
%% allowing the author to define a "short title" to be used in page headers.
\title{Lab 2}

%%
%% The "author" command and its associated commands are used to define
%% the authors and their affiliations.
%% Of note is the shared affiliation of the first two authors, and the
%% "authornote" and "authornotemark" commands
%% used to denote shared contribution to the research.


\author{Allen Jiang}
\email{alljiang@utexas.edu}
\affiliation{%
  \institution{UT Austin, USA}
  \streetaddress{}
  \city{}
  \state{}
  \country{}
  \postcode{}}

\pagenumbering{arabic}
\date{}  % comment this out if you want the date to print



%%
%% This command processes the author and affiliation and title
%% information and builds the first part of the formatted document.
\maketitle

\section{Introduction}
\label{sec:intro}
In this lab, we implement AES-256 and RSA encryption and decryption in C. 
The goal of  this lab is to familiarize ourselves with implementing cryptography algorithms.

We are using CTR block cipher mode for AES-256 encryption and decryption.
The S-box is provided for us, along with a generated key and IV.


% \ignore{comment text} is better than a line comment.
% \ignore{Sometimes background is merged into motivation, and is not required separately.}

\section{Our Architecture}
\label{sec:arch}
The implementation of AES-256 followed the NIST standard for AES-256. 
In the aes.c file, there are a few functions that are used to implement AES-256:
\begin{itemize}
    \item calculate\_round\_keys(): generates all keys needed given a key and the s-box.
    \item encode(): encodes an input given plaintext, a key, round keys, and the s-box.
    \item decode(): decodes an input given ciphertext, a key, round keys, and the s-box. While this function is not used in this lab due to the use of CTR mode, it is still implemented and tested.
\end{itemize}

The implementation of RSA mode is primarily based on the mod\_exponentiation() function. 
This function takes in a base, exponent, and modulus and returns the result of the exponentiation. 
It calculates the result of the exponentiation by using the square-and-multiply algorithm.

To calculate the result, it uses a uint256 struct to store the intermediate results. 
The uint256 struct is constructed by using an array of 8 32-bit unsigned integers.

In addition to this new type, there are also a few functions that were developed to
support operations on the uint256 struct:
\begin{itemize}
    \item compare(): compares two uint256 structs and returns a defined value based on the comparison.
    \item shift\_left(): shifts a uint256 struct left by a given number of bits.
    \item shift\_right(): shifts a uint256 struct right by a given number of bits.
    \item subtract(): subtracts two uint256 structs and returns the result.
    \item mod(): calculates the modulus of two uint256 structs and returns the result.
    \item multiply(): multiplies two uint256 structs and returns the result.
    \item convert\_128\_to\_256(): converts a uint128 struct to a uint256 struct.
    \item convert\_256\_to\_128(): converts a uint256 struct to a uint128 struct.
\end{itemize}

\section{Experimental Results}
\label{sec:results}
The output of the aes.c program passes the given asserts:
\begin{itemize}
    \item assert(memcmp(enc\_buf, ciphertext[0], 32) == 0);
	\item assert(memcmp(decrypted\_text, plaintext[0], 32) == 0);
\end{itemize}

The raw output of the aes.c program is:

\graphicspath{{./}}
\includegraphics*[scale=1]{aes_output.png}

\section{Conclusions}
\label{sec:conclusion}
The output of the aes.c program matches the expected output. 
This indicates that the AES implementation is correct.

One possible addition to this AES implementation is to integrate integrity checks.
This can be done by computing the SHA-256 hash of the plaintext and including it
as a header of the plaintext before encrypting the entire message. 
The receiver can then compute the SHA-256 hash of
the decrypted plaintext without the header and compare it to the included hash header.
If the hashes match, the integrity of the message is verified.

The decrypted output of the ciphertext output of the rsa.c program matches the original
plaintext input. This indicates that the RSA implementation is correct.

To implement integrity checking for RSA, the sender can compute the SHA-256 hash of the
plaintext and encrypt it with the private key. The receiver can then decrypt the hash
with the public key and compare it to the SHA-256 hash of the plaintext. If the hashes
match, the integrity of the message is verified.

\end{document}

\endinput
%%
%% End of file `sample-authordraft.tex'.
